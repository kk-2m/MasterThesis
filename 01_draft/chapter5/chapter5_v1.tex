% History
% 12/06/2024  (岸)	修論下書き用texファイル作成
% 12/12/2024  (岸)	フォントサイズを11pt, 行間を1.5に設定
% コンパイルの仕方
% 		uplatex chapter1_v1.tex
% 		upbibtex chapter1_v1
% 		uplatex chapter1_v1.tex
% 		uplatex chapter1_v1.tex
% 		dvipdfmx chapter1_v1.dvi

% フォントサイズを11ptに設定
\documentclass[a4paper,11pt,nomag]{jsreport}

\usepackage[dvipdfm,truedimen]{geometry}
\geometry{top=22mm,bottom=22mm,left=22mm,right=22mm}
%% jsclasses系で文字サイズ11pt や 12pt をクラスオプションに指定すると,
%% 長さが拡大されるため,nomagオプションを併用している.
%% https://oku.edu.mie-u.ac.jp/~okumura/jsclasses/ のFAQをよく読むこと.

%\usepackage{layout}
%\usepackage[utf8]{inputenc} %不要かも
\usepackage[T1]{fontenc} %utf8フォントエンコーディング指定
\usepackage{lmodern} % 11pt, nomag を使っているので
% CloudLaTeX の場合は下の1行を有効にすること
% \AtBeginDvi{\special{pdf:mapfile ptex-ipaex.map}}
\usepackage{array}
\newcommand{\bhline}[1]{\noalign{\hrule height #1}}  
\newcommand{\bvline}[1]{\vrule width #1}
\renewcommand{\baselinestretch}{1.5} % 教授が赤修正を入れやすいように行間を1.5に設定
\usepackage[singlespacing]{setspace} % 部分的な行間調整パッケージ.表など行間を1.5倍にしたくないところに使う.
\usepackage[subrefformat=parens]{subcaption}
\usepackage[dvipdfmx]{graphicx} % dvipdfmx を前提としている
\usepackage[dvipdfmx]{color}
\usepackage{caption}
\usepackage{subcaption}
\usepackage{multirow}
\usepackage{arydshln}
\usepackage{here} % 図表の位置決め用
\usepackage{amsmath,amssymb}% 数式用
\usepackage{bbm}
\usepackage{bm} % 数式太字
\usepackage{url}      % URL等記載用.\verbより便利
\usepackage{enumerate}
\usepackage{midpage}
% ハイパーリンク
\usepackage[dvipdfmx,breaklinks=true,colorlinks]{hyperref} % dvipdfmxは日本語のときのみかく
\usepackage{pxjahyper} % (u)pLaTeXのときのみかく

% サブキャプションのフォーマットを調整
\renewcommand\thesubfigure{(\alph{subfigure})}
\captionsetup[subfigure]{labelformat=simple, labelsep=space}

\begin{document}
\setcounter{chapter}{5}

\chapter*{結論}

本論文では,限られた赤外線画像を用いた効果的な野生動物監視のための新たな概念として,Infrared Few-shot Open-set Recognition (IFOR)を提案した.
本研究では,最小限の赤外線画像データを用いて,モデルに登録された動物の分類と未登録の動物の識別を実現するIFORフレームワークを開発した.

包括的な分析により得られた主要な知見として,まず特徴抽出器について,ViTは赤外線画像の分類においてCNN(ResNet18)と比較して優れた性能を示し,精度が5.2\%向上した.
これは,ViTの大域的な注意機構が赤外線野生動物画像における特徴抽出において高い有効性を持つことを示唆している.
一方で,基盤モデルであるCLIPはViT-Baseと比較して,有効性を示せなかった.
この結果を踏まえ,今後の研究課題として,テキストエンコーダやプロンプトを用いた学習フレームワークの検証が必要である.
これにより,IFORにおけるテキストと画像のペアを用いた対象学習の有効性を理論的かつ実験的に明らかにすることが可能となる.
転移学習については,ImageNetによる事前学習が,フラクタルベースの事前学習および事前学習なしの場合と比較して一貫して優れた性能を示した.
このことから,自然画像から学習した特徴が赤外線野生動物分類タスクに対して高い汎化性能を持つことを示している.
メタ学習に関しては,提案したメタ学習アプローチが従来の学習手法であるミニバッチ学習と比較して分類精度を11.1\%向上させ,未登録クラスの検出性能であるAUROCを7.7\%改善した.
この結果は,データ数が限られた条件下において,モデルを新規クラスに適応させる際のメタ学習の有効性を実証している.
未登録クラスの多クラス分類においては,k-means損失とBetween-Class損失の導入により既存のPEELERモデルと比較して,登録クラスと未登録クラスを含む全体の分類精度(All Accuracy)を改善した.
これらの実験により,クラスタリングに基づく損失関数が未登録クラスの多クラス分類に対して有効であることが実証された.

また,異なる地域から収集された学習用データセットと評価用データセットを用いることにより,ドメインシフトの条件下におけるモデルの適応性に関する知見を得た.
さらに,学習に用いるサポートデータのショット数の増加がモデルの分類性能に及ぼす影響を検証し,ショット数を増やすことで分類精度が大幅に向上することを明らかにした.
これらの知見は,夜間環境という困難な条件下における野生動物モニタリング技術の進展に貢献するとともに,生態学の研究において直面する限られたデータ条件下でも適用可能な,
効率的かつ汎用性の高い機械学習モデルの開発に向けた基盤となる成果を提供している.

野生動物モニタリングの実用化に向けた今後の課題として,本研究で取り組んだ画像分類に加えて,物体検出手法の開発が挙げられる.
特に,少数の赤外線画像データという制約下においても,画像内の動物領域を高精度に検出可能な手法の開発が急務である.
また,将来の研究における有望な方向性として,IFORフレームワークの性能を潜在的に向上させる多段階転移学習アプローチが挙げられる.
具体的には,ImageNetのような大規模データセットにより事前学習されたモデルを基盤とし,Snapshot Serengetiのような大規模なカメラトラップ画像データセットを用いた中間的なファインチューニングを経て,
最終段階で,ターゲットタスクであるCCTやWCSのような小規模な赤外線画像データセットを用いたメタ学習を行うアプローチである.

\chapter*{結論}

本研究を進めるにあたり,全過程を通じて御助言,御指導を頂きました,滝本裕則准教授に深く感謝致します.
また,本研究及び本論文の執筆において,様々な面で御協力を頂きました,金川明弘教授に感謝致します.
そして,本研究を通して日頃から御討論,御協力頂いた数理情報メディア工学研究室の皆様に御礼申し上げます.
特に,研究に際して助言や協力を頂いた伊藤嵐丸さん,岸本昌子さんに深く感謝申し上げます.

% ここから参考文献bibtexの設定
\bibliographystyle{../kishiIEEEtr}
\bibliography{../references}

\end{document}