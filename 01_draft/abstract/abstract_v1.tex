% History
% 12/06/2024  (岸)	修論下書き用texファイル作成
% 12/12/2024  (岸)	フォントサイズを11pt, 行間を1.5に設定
% コンパイルの仕方
% 		uplatex chapter1_v1.tex
% 		upbibtex chapter1_v1
% 		uplatex chapter1_v1.tex
% 		uplatex chapter1_v1.tex
% 		dvipdfmx chapter1_v1.dvi

% フォントサイズを11ptに設定
\documentclass[a4paper,11pt,nomag]{jsreport}

\usepackage[dvipdfm,truedimen]{geometry}
\geometry{top=22mm,bottom=22mm,left=22mm,right=22mm}
%% jsclasses系で文字サイズ11pt や 12pt をクラスオプションに指定すると,
%% 長さが拡大されるため,nomagオプションを併用している.
%% https://oku.edu.mie-u.ac.jp/~okumura/jsclasses/ のFAQをよく読むこと.

%\usepackage{layout}
%\usepackage[utf8]{inputenc} %不要かも
\usepackage[T1]{fontenc} %utf8フォントエンコーディング指定
\usepackage{lmodern} % 11pt, nomag を使っているので
% CloudLaTeX の場合は下の1行を有効にすること
% \AtBeginDvi{\special{pdf:mapfile ptex-ipaex.map}}
\usepackage{array}
\newcommand{\bhline}[1]{\noalign{\hrule height #1}}  
\newcommand{\bvline}[1]{\vrule width #1}

\usepackage[master]{abst-en}
\usepackage[subrefformat=parens]{subcaption}
\usepackage[dvipdfmx]{graphicx} % dvipdfmx を前提としている
\usepackage[dvipdfmx]{color}
\usepackage{caption}
\usepackage{comment} % コメント用パッケージ
\usepackage{subcaption}
\usepackage{multirow}
\usepackage{arydshln}
\usepackage{here} % 図表の位置決め用
\usepackage{amsmath,amssymb}% 数式用
\usepackage{bbm}
\usepackage{bm} % 数式太字
\usepackage{url}      % URL等記載用.\verbより便利
\usepackage{enumerate}
\usepackage{midpage}
% ハイパーリンク
\usepackage[dvipdfmx,breaklinks=true,colorlinks]{hyperref} % dvipdfmxは日本語のときのみかく
\usepackage{pxjahyper} % (u)pLaTeXのときのみかく

% サブキャプションのフォーマットを調整
\renewcommand\thesubfigure{(\alph{subfigure})}
\captionsetup[subfigure]{labelformat=simple, labelsep=space}

\begin{document}
%\layout
\題目={メタ学習に基づくInfrared Few-shot Open-set Recognitionを考慮した動物分類}
\学籍番号={K623009}           %%%  学籍番号
\英文題目={Animal Classification Considering \\ Infrared Few-shot Open-set Recognition \\ Based on Meta-learning}
\英文氏名={Koki Kishi!}

\newpage
%英文のアブストラクトを書く場合.卒業論文ならコメントアウト.
\begin{abstract}\hspace{0.5zw}
  
\end{abstract}

\begin{comment}
生物多様性は地球の物質や資源循環において不可欠な役割を担っているが、人間活動による生態系の劣化が進行する中で、効果的な生態系モニタリングの重要性が世界的に高まっている。
このモニタリング手法として、カメラトラップを用いた監視は費用対効果が高く、効率的な手法として注目を集めている。
特に夜間撮影が可能な赤外線カメラの活用が期待されているものの、赤外線画像の分類においては色情報の欠如や学習データの不足などの課題が存在する。

本論文では、カメラトラップを使用した野生動物モニタリングのためのInfrared Few-shot Open-set Recognition (IFOR) フレームワークを提案した。
このフレームワークは、限られた赤外線画像データのみを用いて、モデルに登録された動物種の分類と未登録の動物種の検出を同時に実現することを目的としている。
本研究では、赤外線画像に適した特徴抽出器の特定、少数データに効果的な転移学習の有効性検証、メタ学習の有効性の検証を行っている。
さらに、未登録クラスの多クラス分類精度向上を目的として、k-means損失とBetween-Class損失を導入し、クラス内分散の最小化とクラス間分散の最大化を図った。
実験では、南米で撮影されたWCS Camera Trapsデータセットを学習用に、北米で撮影されたCaltech Camera Trapsデータセットを評価用として使用し、地理的条件の差異に起因するドメインシフトを考慮した評価を行った。

本研究の成果は、限られたデータ条件下における効果的な野生動物モニタリングシステムの実現に向けた重要な知見を提供するものであり、生態学研究における実用的な機械学習モデルの開発基盤として活用されることが期待される。
これらの研究成果は、野生動物との持続可能な共生を目指した生態系モニタリング技術の発展に貢献するものである。
\end{comment}

\end{document}