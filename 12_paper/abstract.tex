% /01_draft/abstract/abstract_v2.tex

%英文のアブストラクトを書く場合.(200語程度)
Camera traps are crucial for cost-effective ecosystem monitoring.
In particular, there are high expectations for using infrared cameras to capture night images.
However, there are challenges with classifying infrared images, such as the lack of color information and insufficient training data.

This paper proposes the novel Infrared Few-shot Open-set Recognition (IFOR) framework for wildlife monitoring using camera traps.
The framework considers three key challenges: infrared image classification, limited training data, and unknown species detection.
We evaluated the effectiveness of meta-learning with different feature extractors (CNN and ViT) and transfer learning approaches within the IFOR framework.
Furthermore, to improve the multi-class classification accuracy for unregistered animal species, 
k-means loss and Between-Class loss were introduced to minimize intra-class variance and maximize inter-class variance.
The experiments utilized data collected from different regions for training and evaluation and assessed the model's performance under domain shift conditions.
Experimental results showed that ViT with ImageNet pre-training and meta-learning was effective for IFOR, 
and the proposed loss functions effectively enhanced the classification accuracy of both known and unknown species.

The results of this research provide important insights for realizing effective wildlife monitoring systems under limited data conditions.
Moreover, they are expected to be a foundation for developing practical machine-learning models in ecological research.