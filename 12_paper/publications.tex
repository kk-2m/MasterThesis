% /01_draft/chapter5/chapter5_v2.tex

\chapter*{研究業績}

\subsection*{学術雑誌への掲載論文}
\begin{enumerate}
  \item 北山晃生, 岸孝樹, 古賀荘翠, 滝本裕則, 金川明弘, ``室内での見守り実現に向けた全方位画像に対する複数人物追跡,'' 日本福祉工学会誌, Vol.~26, No.~2, pp.~2-8(7ページ), 2024年11月.
  \item Koki Kishi, Ranmaru Ito, Sulfayanti Faharuddin Situju, Hironori Takimoto, and Akihiro Kanagawa, ``Animal Classification Considering Infrared Few-shot Open-set Recognition,'' \mbox{\textit{Cogent Engineering}}, (Accepted), 2025年1月.
\end{enumerate}

\subsection*{国際会議での研究発表}
\begin{enumerate}
  \item ◎Koki Kishi, Masako Kishimoto, Sulfayanti Faharuddin Situju, Hironori Takimoto and Akihiro Kanagawa, ``Few-Shot Learning for CNN-based Animal Classification in Camera Traps using an Infrared Camera,'' In \textit{Proceedings of the 10th IIAE International Conference on Intelligent Systems and Image Processing 2023 (ICISIP2023)}, pp.~11-17(7ページ), 2023年9月.
  \item ◎Koki Kishi, Ranmaru Ito, Sulfayanti Faharuddin Situju, Hironori Takimoto and Akihiro Kanagawa, ``Few-Shot Learning for Animal Classification in Camera Traps using an Infrared Camera,'' In \textit{Proceedings of the SICE Festival 2024 with Annual Conference (SICE FES 2024)}, WeBT7.7, pp.~420-423 (4ページ), 2024年8月.
\end{enumerate}

\subsection*{国内学会(全国レベル)での研究発表}
\begin{enumerate}
  \item ◎古賀荘翠, 北山晃生, 岸孝樹, 滝本裕則, 金川明弘, ``複数物体追跡におけるIDスイッチ抑制のためのMotion SORTの提案,'' 第28回 パターン計測シンポジウム, PM108\_04(4ページ), 2023年11月.
  \item ◎岸孝樹, 伊藤嵐丸, 植田諒大, Sulfayanti Faharuddin Situju, 滝本裕則, ``Infrared Few-shot Open-set Recognitionを考慮したクラスタリングとメタ学習による動物分類,'' 第29回 パターン計測シンポジウム, PM109\_02(8ページ), 2024年11月.
\end{enumerate}

\subsection*{国内学会(支部・県レベル)での研究発表}
\begin{enumerate}
  \item ◎岸孝樹, 伊藤嵐丸, 滝本裕則, 金川明弘, ``赤外線カメラトラップにおける動物分類のための少数データ学習,'' 第26回IEEE広島支部学生シンポジウム (IEEE HISS 26th) 論文集, TP-A-25, pp.~91-94(4ページ), 2024年11月.
\end{enumerate}

\subsection*{受賞}
\begin{enumerate}
  \item The 10th IIAE International Conference on Intelligent Systems and Image Processing 2023 Best Paper Award, 2023年9月(対象は上記の国際会議での研究発表(1)).
  \item 計測自動制御学会 計測部門 パターン計測部会 令和6年度優秀論文賞, 2024年11月(対象は上記の国内学会(全国レベル)での研究発表(2)).
  \item IEEE広島支部学生シンポジウム (HISS) 優秀研究賞, 2024年11月(対象は上記の国内学会(支部・県レベル)での研究発表(1)).
\end{enumerate}